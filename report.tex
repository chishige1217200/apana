\documentclass[autodetect-engine,dvi=dvipdfmx,ja=standard,
               a4j,11pt]{bxjsarticle}

\RequirePackage{geometry}
\geometry{reset,a4paper}
\geometry{hmargin=25truemm,top=25truemm,bottom=25truemm,footskip=10truemm}
%\geometry{showframe} % 本文の"枠"を確認したければ,コメントアウト

% 設定:図の挿入
% http://www.edu.cs.okayama-u.ac.jp/info/tool_guide/tex.html#graphicx
\usepackage{graphicx}

% 設定:ソースコードの挿入
% http://www.edu.cs.okayama-u.ac.jp/info/tool_guide/tex.html#fancyvrb
\usepackage{fancyvrb}
\renewcommand{\theFancyVerbLine}{\texttt{\footnotesize{\arabic{FancyVerbLine}:}}}

\title{応用解析レポート}

\author{氏名: 重近 大智 (SHIGECHIKA, Daichi) \\
        学生番号: 09501527}

\date{提出日: 2020年08月04日 \\
      締切日: 2020年08月06日 \\}

\begin{document}

\maketitle{\footnotesize \tableofcontents \newpage}
%%%%%%%%%%%%%%%%%%%%%%%%%%%%%%%%%%%%%%%%%%%%%%%%%%%%%%%%%%%%%%%%%%%%%%
\section{概要}
%%%%%%%%%%%%%%%%%%%%%%%%%%%%%%%%%%%%%%%%%%%%%%%%%%%%%%%%%%%%%%%%%%%%%%
本レポートでは台形積分及びシンプソン積分を行うC言語のプログラムを作成し,その積分の結果,絶対誤差と考察を報告する.
なお各積分において,分割数が$n=2,4,8,16$の4通りの場合を計算する.

%%%%%%%%%%%%%%%%%%%%%%%%%%%%%%%%%%%%%%%%%%%%%%%%%%%%%%%%%%%%%%%%%%%%%%
\section{プログラムの説明と計算結果}
%%%%%%%%%%%%%%%%%%%%%%%%%%%%%%%%%%%%%%%%%%%%%%%%%%%%%%%%%%%%%%%%%%%%%%
計算結果と絶対誤差は,表\ref{tab:result}に示す.また,計算に使用したプログラムは\ref{sec:makep}章に示す.

簡単にプログラムの各関数の役割について説明する.
プログラムの9--17行目で宣言されている\verb|f()|関数で,今回の被積分関数である$1/x^3$の計算を行う.
19--41行目で宣言されている\verb|trape()|関数は,台形積分の計算を行う.
43--67行目で宣言されている\verb|simp()|関数は,シンプソン積分の計算を行う.
69--84行目で宣言されている\verb|main()|関数では,積分方法と分割数を\verb|printf()|関数により表示し,使用者に入力を求める.積分方法で$1$を選択すると台形積分,$2$を選択するとシンプソン積分となる.


\begin{table}[b]
\centering

\caption{計算結果と絶対誤差}
\label{tab:result}
\begin{tabular}{|l|l|l|l|}
\hline
積分方法 & 分割数$n$ & 計算結果 & 絶対誤差 \\
\hline
台形積分 & 2 & 0.429398 & 0.054398 \\
& 4 & 0.389346 & 0.014346 \\
& 8 & 0.378642 & 0.003642 \\
& 16 & 0.375914 & 0.000914 \\
\hline
シンプソン積分 & 2 & 0.385031 & 0.010031 \\
& 4 & 0.375996 & 0.000996 \\
& 8 & 0.375074 & 0.000074 \\
& 16 & 0.375005 & 0.000005 \\
\hline
\end{tabular}
\end{table}
%%%%%%%%%%%%%%%%%%%%%%%%%%%%%%%%%%%%%%%%%%%%%%%%%%%%%%%%%%%%%%%%%%%%%%
\section{考察}
%%%%%%%%%%%%%%%%%%%%%%%%%%%%%%%%%%%%%%%%%%%%%%%%%%%%%%%%%%%%%%%%%%%%%%
今回の積分の真値は$0.375$である.台形積分では分割数が$2$の場合に,大きな絶対誤差がある.それに対し,シンプソン積分では分割数が$2$の場合でも,誤差は約$0.01$と少ない.分割数を増やしていくといずれの場合も,台形積分に比べシンプソン積分の方が絶対誤差が小さい.$16$分割したシンプソン積分では小数第5位までの絶対誤差がなくなり,かなり精度の高い計算が行われている.以上の結果からシンプソン積分の方が台形積分に比べ,速く真値に収束することが分かる.


%%%%%%%%%%%%%%%%%%%%%%%%%%%%%%%%%%%%%%%%%%%%%%%%%%%%%%%%%%%%%%%%%%%%%%
\section{作成したプログラム}\label{sec:makep}
%%%%%%%%%%%%%%%%%%%%%%%%%%%%%%%%%%%%%%%%%%%%%%%%%%%%%%%%%%%%%%%%%%%%%%
作成したプログラムを以下に添付する.プログラムは84行からなる.
\begin{Verbatim}[fontsize=\small, baselinestretch=0.8]
     1	#include <stdio.h>
     2	#include <math.h>
     3	#include <stdlib.h>
     4	
     5	double a = 1;
     6	double b = 2;
     7	double t_value = 0.375;
     8	
     9	double f(double x)
    10	{
    11	  double y;
    12	
    13	  x = x * x * x;
    14	  y = 1 / x;
    15	
    16	  return y;
    17	}
    18	
    19	void trape(int part)
    20	{
    21	  int i;
    22	  double h;
    23	  double x;
    24	  double y_l;
    25	  double y_r;
    26	  double S = 0;
    27	
    28	  h = fabs((b - a) / part);
    29	  x = a;
    30	
    31	  for(i = 0; i < part; i++)
    32	    {
    33	      y_l = f(x);
    34	      y_r = f(x + h);
    35	      S = S + ((y_l + y_r) * h) / 2;
    36	      x = x + h;
    37	    }
    38	  fprintf(stderr, "\n");
    39	  printf("Trapezoidal Answer:%f\n", S);
    40	  printf("The difference    :%f\n", fabs(S - t_value));
    41	}
    42	
    43	void simp(int part)
    44	{
    45	  int i;
    46	  double h;
    47	  double x;
    48	  double y_l;
    49	  double y_c;
    50	  double y_r;
    51	  double S = 0;
    52	
    53	  h = fabs((b - a) / part);
    54	  x = a;
    55	
    56	  for(i = 0; x < b; i++)
    57	    {
    58	      y_l = f(x);
    59	      y_c = f(x + h);
    60	      y_r = f(x + 2 * h);
    61	      S = S + h * (y_l + 4 * y_c + y_r) / 3;
    62	      x = x + 2 * h;
    63	    }
    64	  fprintf(stderr, "\n");
    65	  printf("Simpson's Answer:  %f\n", S);
    66	  printf("The difference  :  %f\n", fabs(S - t_value));
    67	}
    68	
    69	int main(void)
    70	{
    71	  int mode = 0;
    72	  int part = 2;
    73	
    74	  fprintf(stderr, "Please select integral rule. (1:Trapezoidal rule 2:Simpson's ru
le)\n");
    75	  scanf("%d", &mode);
    76	
    77	  fprintf(stderr, "Please input number of partitions.\n");
    78	  scanf("%d", &part);
    79	
    80	  if(mode == 1) trape(part);
    81	  if(mode == 2) simp(part);
    82	
    83	  return 0;
    84	}
\end{Verbatim}

\end{document}
